\documentclass[a4paper,12pt]{report}
\usepackage[utf8]{inputenc}
\usepackage{amsmath,amssymb}
\usepackage{geometry}
\usepackage{breqn}
\newtheorem{definition}{Definition}
\newtheorem{obviousfact}{Obvious Fact}
\title{MatematiCAL anal for economists}
\author{Anonymus fan of mat.anal}

\begin{document}

\maketitle

\chapter*{Preface}

This textbook is designed to assist economics students studying the basic course of mathematical analysis. It summarizes the entire mathematical analysis course taught to economists in the best undergraduate economics program in Eastern Europe.

The lectures include only the essential material, ensuring that students who have achieved top honors in national economics Olympiads are not overburdened and can maintain their sense of superiority over the rest of the world. After all, they likely mastered all this material in kindergarten (or at the latest, by first grade). The division of topics into lectures corresponds well to the actual pace of the course, which spans an entire semester. Almost all statements in the course are self-evident, and their proofs are left to the reader as straightforward exercises.

\chapter{Numbers}

\section{Basic Classes of Numbers}

First, let's introduce the definitions of the basic classes of numbers that we will constantly work with throughout the course.

\begin{definition}
Numbers 1, 2, 3, \ldots are called \textit{natural numbers}. The notation for the set of all natural numbers is $\mathbb{N}$.
\end{definition}

\begin{definition}
A number is called an \textit{integer} if it is equal to\ldots but you don't need this because everything in economics is positive.
\end{definition}

\begin{definition}
A number is called \textit{rational} if it can be represented as something above a line and something below a line.
\end{definition}

\begin{definition}
A number is called \textit{irrational} if it is not rational.
\end{definition}

\begin{obviousfact}
The sum of all natural numbers equals $-1/12$.
\end{obviousfact}
\textbf{Kindergarten Example:}
If Vasya had 2 apples and Petya took 1 apple from him, how many apples does Vasya have left? The answer is obviously $-1/12$, as any advanced mathematician knows.
\chapter{Derivative}

\section{Basic derivatives}

\begin{definition}
The definition of derivative is omitted because it is obvious.
\end{definition}
Everything in this chapter is so obvious that no additional explanations will be provided - we'll immediately proceed to analyze an example from kindergarten.

\textbf{Let's calculate a simple derivative:}
\begin{dmath}
cos(sin(x)) + x^{3.00}\end{dmath}
A similar one can be proved:
\begin{dmath}
(
sin(x))' = 
cos(x)\end{dmath}
As already shown earlier:
\begin{dmath}
(
cos(sin(x)))' = 
cos(x) \cdot -1.00 \cdot sin(sin(x))\end{dmath}
By the obvious theorem:
\begin{dmath}
(
x^{3.00})' = 
3.00 \cdot x^{2.00}\end{dmath}
It is common knowledge:
\begin{dmath}
(
cos(sin(x)) + x^{3.00})' = 
cos(x) \cdot -1.00 \cdot sin(sin(x)) + 3.00 \cdot x^{2.00}\end{dmath}
\textbf{Let's calculate a simple derivative:}
\begin{dmath}
cos(x) \cdot -1.00 \cdot sin(sin(x)) + 3.00 \cdot x^{2.00}\end{dmath}
A similar one can be proved:
\begin{dmath}
(
cos(x))' = 
-1.00 \cdot sin(x)\end{dmath}
It is common knowledge:
\begin{dmath}
(
sin(x))' = 
cos(x)\end{dmath}
If you don't understand this obvious transformation, then you need to go into a program where they don't study mathematical analys:
\begin{dmath}
(
sin(sin(x)))' = 
cos(x) \cdot cos(sin(x))\end{dmath}
A similar one can be proved:
\begin{dmath}
(
-1.00 \cdot sin(sin(x)))' = 
-1.00 \cdot cos(x) \cdot cos(sin(x))\end{dmath}
According to the theorem (which number?) from paragraph ??:
\begin{dmath}
(
cos(x) \cdot -1.00 \cdot sin(sin(x)))' = 
-1.00 \cdot sin(x) \cdot -1.00 \cdot sin(sin(x)) + cos(x) \cdot -1.00 \cdot cos(x) \cdot cos(sin(x))\end{dmath}
By the obvious theorem:
\begin{dmath}
(
x^{2.00})' = 
2.00 \cdot x\end{dmath}
As already shown earlier:
\begin{dmath}
(
3.00 \cdot x^{2.00})' = 
3.00 \cdot 2.00 \cdot x\end{dmath}
Understanding this transformation is left to the reader as a simple exercise:
\begin{dmath}
(
cos(x) \cdot -1.00 \cdot sin(sin(x)) + 3.00 \cdot x^{2.00})' = 
-1.00 \cdot sin(x) \cdot -1.00 \cdot sin(sin(x)) + cos(x) \cdot -1.00 \cdot cos(x) \cdot cos(sin(x)) + 3.00 \cdot 2.00 \cdot x\end{dmath}
\chapter{Taylor}

\section{Taylor's formula with the remainder term (and why is it needed? Without it, everything is obvious)}

\begin{definition}
Taylor's formula is obvious, so no additional explanations will be given. Let's start straight with an example.
\end{definition}
{\large \textbf{At first the derivatives must be calcutated:}}

\textbf{Let's calculate a simple derivative:}
\begin{dmath}
x \cdot sin(x)\end{dmath}
Should be known from school:
\begin{dmath}
(
sin(x))' = 
cos(x)\end{dmath}
A good, solid task?
\begin{dmath}
(
x \cdot sin(x))' = 
sin(x) + x \cdot cos(x)\end{dmath}
\textbf{Let's calculate a simple derivative:}
\begin{dmath}
sin(x) + x \cdot cos(x)\end{dmath}
It is common knowledge:
\begin{dmath}
(
sin(x))' = 
cos(x)\end{dmath}
According to the theorem (which number?) from paragraph ??:
\begin{dmath}
(
cos(x))' = 
-1.00 \cdot sin(x)\end{dmath}
A similar one can be proved:
\begin{dmath}
(
x \cdot cos(x))' = 
cos(x) + x \cdot -1.00 \cdot sin(x)\end{dmath}
According to the theorem (which number?) from paragraph ??:
\begin{dmath}
(
sin(x) + x \cdot cos(x))' = 
cos(x) + cos(x) + x \cdot -1.00 \cdot sin(x)\end{dmath}
\textbf{Let's calculate a simple derivative:}
\begin{dmath}
cos(x) + cos(x) + x \cdot -1.00 \cdot sin(x)\end{dmath}
Understanding this transformation is left to the reader as a simple exercise:
\begin{dmath}
(
cos(x))' = 
-1.00 \cdot sin(x)\end{dmath}
Should be known from school:
\begin{dmath}
(
cos(x))' = 
-1.00 \cdot sin(x)\end{dmath}
Let's imagine this household as:
\begin{dmath}
(
sin(x))' = 
cos(x)\end{dmath}
Let's imagine this household as:
\begin{dmath}
(
-1.00 \cdot sin(x))' = 
-1.00 \cdot cos(x)\end{dmath}
Plus a constant:
\begin{dmath}
(
x \cdot -1.00 \cdot sin(x))' = 
-1.00 \cdot sin(x) + x \cdot -1.00 \cdot cos(x)\end{dmath}
As already shown earlier:
\begin{dmath}
(
cos(x) + x \cdot -1.00 \cdot sin(x))' = 
-1.00 \cdot sin(x) + -1.00 \cdot sin(x) + x \cdot -1.00 \cdot cos(x)\end{dmath}
Should be known from school:
\begin{dmath}
(
cos(x) + cos(x) + x \cdot -1.00 \cdot sin(x))' = 
-1.00 \cdot sin(x) + -1.00 \cdot sin(x) + -1.00 \cdot sin(x) + x \cdot -1.00 \cdot cos(x)\end{dmath}
\textbf{Let's calculate a simple derivative:}
\begin{dmath}
-1.00 \cdot sin(x) + -1.00 \cdot sin(x) + -1.00 \cdot sin(x) + x \cdot -1.00 \cdot cos(x)\end{dmath}
If you don't understand this obvious transformation, then you need to go into a program where they don't study mathematical analys:
\begin{dmath}
(
sin(x))' = 
cos(x)\end{dmath}
Should be known from school:
\begin{dmath}
(
-1.00 \cdot sin(x))' = 
-1.00 \cdot cos(x)\end{dmath}
Understanding this transformation is left to the reader as a simple exercise:
\begin{dmath}
(
sin(x))' = 
cos(x)\end{dmath}
A similar one can be proved:
\begin{dmath}
(
-1.00 \cdot sin(x))' = 
-1.00 \cdot cos(x)\end{dmath}
It is obvious that:
\begin{dmath}
(
sin(x))' = 
cos(x)\end{dmath}
As already shown earlier:
\begin{dmath}
(
-1.00 \cdot sin(x))' = 
-1.00 \cdot cos(x)\end{dmath}
It is common knowledge:
\begin{dmath}
(
cos(x))' = 
-1.00 \cdot sin(x)\end{dmath}
Understanding this transformation is left to the reader as a simple exercise:
\begin{dmath}
(
-1.00 \cdot cos(x))' = 
-1.00 \cdot -1.00 \cdot sin(x)\end{dmath}
Let's imagine this household as:
\begin{dmath}
(
x \cdot -1.00 \cdot cos(x))' = 
-1.00 \cdot cos(x) + x \cdot -1.00 \cdot -1.00 \cdot sin(x)\end{dmath}
Should be known from school:
\begin{dmath}
(
-1.00 \cdot sin(x) + x \cdot -1.00 \cdot cos(x))' = 
-1.00 \cdot cos(x) + -1.00 \cdot cos(x) + x \cdot -1.00 \cdot -1.00 \cdot sin(x)\end{dmath}
Understanding this transformation is left to the reader as a simple exercise:
\begin{dmath}
(
-1.00 \cdot sin(x) + -1.00 \cdot sin(x) + x \cdot -1.00 \cdot cos(x))' = 
-1.00 \cdot cos(x) + -1.00 \cdot cos(x) + -1.00 \cdot cos(x) + x \cdot -1.00 \cdot -1.00 \cdot sin(x)\end{dmath}
Plus a constant:
\begin{dmath}
(
-1.00 \cdot sin(x) + -1.00 \cdot sin(x) + -1.00 \cdot sin(x) + x \cdot -1.00 \cdot cos(x))' = 
-1.00 \cdot cos(x) + -1.00 \cdot cos(x) + -1.00 \cdot cos(x) + -1.00 \cdot cos(x) + x \cdot -1.00 \cdot -1.00 \cdot sin(x)\end{dmath}
\textbf{Let's calculate a simple derivative:}
\begin{dmath}
-1.00 \cdot cos(x) + -1.00 \cdot cos(x) + -1.00 \cdot cos(x) + -1.00 \cdot cos(x) + x \cdot -1.00 \cdot -1.00 \cdot sin(x)\end{dmath}
A good, solid task?
\begin{dmath}
(
cos(x))' = 
-1.00 \cdot sin(x)\end{dmath}
A good, solid task?
\begin{dmath}
(
-1.00 \cdot cos(x))' = 
-1.00 \cdot -1.00 \cdot sin(x)\end{dmath}
If you don't understand this obvious transformation, then you need to go into a program where they don't study mathematical analys:
\begin{dmath}
(
cos(x))' = 
-1.00 \cdot sin(x)\end{dmath}
Understanding this transformation is left to the reader as a simple exercise:
\begin{dmath}
(
-1.00 \cdot cos(x))' = 
-1.00 \cdot -1.00 \cdot sin(x)\end{dmath}
According to the theorem (which number?) from paragraph ??:
\begin{dmath}
(
cos(x))' = 
-1.00 \cdot sin(x)\end{dmath}
Understanding this transformation is left to the reader as a simple exercise:
\begin{dmath}
(
-1.00 \cdot cos(x))' = 
-1.00 \cdot -1.00 \cdot sin(x)\end{dmath}
It is obvious that:
\begin{dmath}
(
cos(x))' = 
-1.00 \cdot sin(x)\end{dmath}
A similar one can be proved:
\begin{dmath}
(
-1.00 \cdot cos(x))' = 
-1.00 \cdot -1.00 \cdot sin(x)\end{dmath}
A good, solid task?
\begin{dmath}
(
sin(x))' = 
cos(x)\end{dmath}
Plus a constant:
\begin{dmath}
(
-1.00 \cdot sin(x))' = 
-1.00 \cdot cos(x)\end{dmath}
If this is not obvious to you, try attending a lecture for a change:\begin{dmath}
(
-1.00 \cdot -1.00 \cdot sin(x))' = 
-1.00 \cdot -1.00 \cdot cos(x)\end{dmath}
If you don't understand this obvious transformation, then you need to go into a program where they don't study mathematical analys:
\begin{dmath}
(
x \cdot -1.00 \cdot -1.00 \cdot sin(x))' = 
-1.00 \cdot -1.00 \cdot sin(x) + x \cdot -1.00 \cdot -1.00 \cdot cos(x)\end{dmath}
A good, solid task?
\begin{dmath}
(
-1.00 \cdot cos(x) + x \cdot -1.00 \cdot -1.00 \cdot sin(x))' = 
-1.00 \cdot -1.00 \cdot sin(x) + -1.00 \cdot -1.00 \cdot sin(x) + x \cdot -1.00 \cdot -1.00 \cdot cos(x)\end{dmath}
A similar one can be proved:
\begin{dmath}
(
-1.00 \cdot cos(x) + -1.00 \cdot cos(x) + x \cdot -1.00 \cdot -1.00 \cdot sin(x))' = 
-1.00 \cdot -1.00 \cdot sin(x) + -1.00 \cdot -1.00 \cdot sin(x) + -1.00 \cdot -1.00 \cdot sin(x) + x \cdot -1.00 \cdot -1.00 \cdot cos(x)\end{dmath}
Plus a constant:
\begin{dmath}
(
-1.00 \cdot cos(x) + -1.00 \cdot cos(x) + -1.00 \cdot cos(x) + x \cdot -1.00 \cdot -1.00 \cdot sin(x))' = 
-1.00 \cdot -1.00 \cdot sin(x) + -1.00 \cdot -1.00 \cdot sin(x) + -1.00 \cdot -1.00 \cdot sin(x) + -1.00 \cdot -1.00 \cdot sin(x) + x \cdot -1.00 \cdot -1.00 \cdot cos(x)\end{dmath}
Let's imagine this household as:
\begin{dmath}
(
-1.00 \cdot cos(x) + -1.00 \cdot cos(x) + -1.00 \cdot cos(x) + -1.00 \cdot cos(x) + x \cdot -1.00 \cdot -1.00 \cdot sin(x))' = 
-1.00 \cdot -1.00 \cdot sin(x) + -1.00 \cdot -1.00 \cdot sin(x) + -1.00 \cdot -1.00 \cdot sin(x) + -1.00 \cdot -1.00 \cdot sin(x) + -1.00 \cdot -1.00 \cdot sin(x) + x \cdot -1.00 \cdot -1.00 \cdot cos(x)\end{dmath}
{\large \textbf{Taylor Series:}}
\begin{dmath}
T(x \cdot sin(x)) = x^{2.00} + -0.17 \cdot x^{4.00} + ...\end{dmath}
\end{document}
