\documentclass[a4paper,12pt]{report}
\usepackage[utf8]{inputenc}
\usepackage{amsmath,amssymb}
\usepackage{geometry}
\usepackage{breqn}
\newtheorem{definition}{Definition}
\newtheorem{obviousfact}{Obvious Fact}
\title{Mathematical anal for economists}
\author{Anonymus fan of mat.anal}

\begin{document}

\maketitle

\chapter*{Preface}

This textbook is designed to assist economics students studying the basic course of mathematical analysis. It summarizes the entire mathematical analysis course taught to economists in the best undergraduate economics program in Eastern Europe.

The lectures include only the essential material, ensuring that students who have achieved top honors in national economics Olympiads are not overburdened and can maintain their sense of superiority over the rest of the world. After all, they likely mastered all this material in kindergarten (or at the latest, by first grade). The division of topics into lectures corresponds well to the actual pace of the course, which spans an entire semester. Almost all statements in the course are self-evident, and their proofs are left to the reader as straightforward exercises.

\chapter{Numbers}

\section{Basic Classes of Numbers}

First, let's introduce the definitions of the basic classes of numbers that we will constantly work with throughout the course.

\begin{definition}
Numbers 1, 2, 3, \ldots are called \textit{natural numbers}. The notation for the set of all natural numbers is $\mathbb{N}$.
\end{definition}

\begin{definition}
A number is called an \textit{integer} if it is equal to\ldots but you don't need this because everything in economics is positive.
\end{definition}

\begin{definition}
A number is called \textit{rational} if it can be represented as something above a line and something below a line.
\end{definition}

\begin{definition}
A number is called \textit{irrational} if it is not rational.
\end{definition}

\begin{obviousfact}
The sum of all natural numbers equals $-1/12$.
\end{obviousfact}
\textbf{Kindergarten Example:}
If Vasya had 2 apples and Petya took 1 apple from him, how many apples does Vasya have left? The answer is obviously $-1/12$, as any advanced mathematician knows.
\chapter{Derivative}

\section{Basic derivatives}

\begin{definition}
The definition of derivative is omitted because it is obvious.
\end{definition}
Everything in this chapter is so obvious that no additional explanations will be provided - we'll immediately proceed to analyze an example from kindergarten.
\textbf{Let's calculate a simple derivative:}
\begin{dmath}
sin(2.00 \cdot x) + cos(2.00 \cdot x) + tg(2.00 \cdot x) + ctg(2.00 \cdot x) + sh(2.00 \cdot x) + ch(2.00 \cdot x) + ln(2.00 \cdot x) + th(2.00 \cdot x) + cth(2.00 \cdot x)\end{dmath}
Understanding this transformation is left to the reader as a simple exercise:
\begin{dmath}
(
2.00 \cdot x)' = 
2.00\end{dmath}
It is common knowledge:
\begin{dmath}
(
sin(2.00 \cdot x))' = 
2.00 \cdot cos(2.00 \cdot x)\end{dmath}
Understanding this transformation is left to the reader as a simple exercise:
\begin{dmath}
(
2.00 \cdot x)' = 
2.00\end{dmath}
According to the theorem (which number?) from paragraph ??:
\begin{dmath}
(
cos(2.00 \cdot x))' = 
2.00 \cdot -1.00 \cdot sin(2.00 \cdot x)\end{dmath}
It is obvious that:
\begin{dmath}
(
sin(2.00 \cdot x) + cos(2.00 \cdot x))' = 
2.00 \cdot cos(2.00 \cdot x) + 2.00 \cdot -1.00 \cdot sin(2.00 \cdot x)\end{dmath}
Understanding this transformation is left to the reader as a simple exercise:
\begin{dmath}
(
2.00 \cdot x)' = 
2.00\end{dmath}
 A similar one can be proved:
\begin{dmath}
(
tg(2.00 \cdot x))' = 
\frac{2.00}{cos(2.00 \cdot x)^{2.00}}\end{dmath}
It is obvious that:
\begin{dmath}
(
sin(2.00 \cdot x) + cos(2.00 \cdot x) + tg(2.00 \cdot x))' = 
2.00 \cdot cos(2.00 \cdot x) + 2.00 \cdot -1.00 \cdot sin(2.00 \cdot x) + \frac{2.00}{cos(2.00 \cdot x)^{2.00}}\end{dmath}
Understanding this transformation is left to the reader as a simple exercise:
\begin{dmath}
(
2.00 \cdot x)' = 
2.00\end{dmath}
If this is not obvious to you, try attending a lecture for a change:\begin{dmath}
(
ctg(2.00 \cdot x))' = 
\frac{2.00}{sin(2.00 \cdot x)^{2.00}} \cdot -1.00\end{dmath}
It is obvious that:
\begin{dmath}
(
sin(2.00 \cdot x) + cos(2.00 \cdot x) + tg(2.00 \cdot x) + ctg(2.00 \cdot x))' = 
2.00 \cdot cos(2.00 \cdot x) + 2.00 \cdot -1.00 \cdot sin(2.00 \cdot x) + \frac{2.00}{cos(2.00 \cdot x)^{2.00}} + \frac{2.00}{sin(2.00 \cdot x)^{2.00}} \cdot -1.00\end{dmath}
Understanding this transformation is left to the reader as a simple exercise:
\begin{dmath}
(
2.00 \cdot x)' = 
2.00\end{dmath}
Let's imagine this household as:
\begin{dmath}
(
sh(2.00 \cdot x))' = 
2.00 \cdot ch(2.00 \cdot x)\end{dmath}
It is obvious that:
\begin{dmath}
(
sin(2.00 \cdot x) + cos(2.00 \cdot x) + tg(2.00 \cdot x) + ctg(2.00 \cdot x) + sh(2.00 \cdot x))' = 
2.00 \cdot cos(2.00 \cdot x) + 2.00 \cdot -1.00 \cdot sin(2.00 \cdot x) + \frac{2.00}{cos(2.00 \cdot x)^{2.00}} + \frac{2.00}{sin(2.00 \cdot x)^{2.00}} \cdot -1.00 + 2.00 \cdot ch(2.00 \cdot x)\end{dmath}
Understanding this transformation is left to the reader as a simple exercise:
\begin{dmath}
(
2.00 \cdot x)' = 
2.00\end{dmath}
Plus a constant:
\begin{dmath}
(
ch(2.00 \cdot x))' = 
2.00 \cdot sh(2.00 \cdot x)\end{dmath}
It is obvious that:
\begin{dmath}
(
sin(2.00 \cdot x) + cos(2.00 \cdot x) + tg(2.00 \cdot x) + ctg(2.00 \cdot x) + sh(2.00 \cdot x) + ch(2.00 \cdot x))' = 
2.00 \cdot cos(2.00 \cdot x) + 2.00 \cdot -1.00 \cdot sin(2.00 \cdot x) + \frac{2.00}{cos(2.00 \cdot x)^{2.00}} + \frac{2.00}{sin(2.00 \cdot x)^{2.00}} \cdot -1.00 + 2.00 \cdot ch(2.00 \cdot x) + 2.00 \cdot sh(2.00 \cdot x)\end{dmath}
Understanding this transformation is left to the reader as a simple exercise:
\begin{dmath}
(
2.00 \cdot x)' = 
2.00\end{dmath}
As already shown earlier:
\begin{dmath}
(
ln(2.00 \cdot x))' = 
2.00 \cdot \frac{1.00}{2.00 \cdot x}\end{dmath}
It is obvious that:
\begin{dmath}
(
sin(2.00 \cdot x) + cos(2.00 \cdot x) + tg(2.00 \cdot x) + ctg(2.00 \cdot x) + sh(2.00 \cdot x) + ch(2.00 \cdot x) + ln(2.00 \cdot x))' = 
2.00 \cdot cos(2.00 \cdot x) + 2.00 \cdot -1.00 \cdot sin(2.00 \cdot x) + \frac{2.00}{cos(2.00 \cdot x)^{2.00}} + \frac{2.00}{sin(2.00 \cdot x)^{2.00}} \cdot -1.00 + 2.00 \cdot ch(2.00 \cdot x) + 2.00 \cdot sh(2.00 \cdot x) + 2.00 \cdot \frac{1.00}{2.00 \cdot x}\end{dmath}
Understanding this transformation is left to the reader as a simple exercise:
\begin{dmath}
(
2.00 \cdot x)' = 
2.00\end{dmath}
A good, solid task?
\begin{dmath}
(
th(2.00 \cdot x))' = 
\frac{2.00}{ch(2.00 \cdot x)^{2.00}}\end{dmath}
It is obvious that:
\begin{dmath}
(
sin(2.00 \cdot x) + cos(2.00 \cdot x) + tg(2.00 \cdot x) + ctg(2.00 \cdot x) + sh(2.00 \cdot x) + ch(2.00 \cdot x) + ln(2.00 \cdot x) + th(2.00 \cdot x))' = 
2.00 \cdot cos(2.00 \cdot x) + 2.00 \cdot -1.00 \cdot sin(2.00 \cdot x) + \frac{2.00}{cos(2.00 \cdot x)^{2.00}} + \frac{2.00}{sin(2.00 \cdot x)^{2.00}} \cdot -1.00 + 2.00 \cdot ch(2.00 \cdot x) + 2.00 \cdot sh(2.00 \cdot x) + 2.00 \cdot \frac{1.00}{2.00 \cdot x} + \frac{2.00}{ch(2.00 \cdot x)^{2.00}}\end{dmath}
Understanding this transformation is left to the reader as a simple exercise:
\begin{dmath}
(
2.00 \cdot x)' = 
2.00\end{dmath}
If you don't understand this obvious transformation, then you need to go into a program where they don't study mathematical analys:
\begin{dmath}
(
cth(2.00 \cdot x))' = 
\frac{2.00}{sh(2.00 \cdot x)^{2.00}} \cdot -1.00\end{dmath}
It is obvious that:
\begin{dmath}
(
sin(2.00 \cdot x) + cos(2.00 \cdot x) + tg(2.00 \cdot x) + ctg(2.00 \cdot x) + sh(2.00 \cdot x) + ch(2.00 \cdot x) + ln(2.00 \cdot x) + th(2.00 \cdot x) + cth(2.00 \cdot x))' = 
2.00 \cdot cos(2.00 \cdot x) + 2.00 \cdot -1.00 \cdot sin(2.00 \cdot x) + \frac{2.00}{cos(2.00 \cdot x)^{2.00}} + \frac{2.00}{sin(2.00 \cdot x)^{2.00}} \cdot -1.00 + 2.00 \cdot ch(2.00 \cdot x) + 2.00 \cdot sh(2.00 \cdot x) + 2.00 \cdot \frac{1.00}{2.00 \cdot x} + \frac{2.00}{ch(2.00 \cdot x)^{2.00}} + \frac{2.00}{sh(2.00 \cdot x)^{2.00}} \cdot -1.00\end{dmath}
\end{document}
